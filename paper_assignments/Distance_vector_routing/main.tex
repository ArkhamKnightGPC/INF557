\documentclass[]{article}
\pagenumbering{gobble}
\usepackage[a4paper]{geometry}

\usepackage{pgfplots}
\usepackage{tcolorbox}
\usepackage{circuitikz}
\usepackage{amsmath}
\usepackage{pgfplotstable}
\pgfplotsset{compat=1.18}

\usepackage{tikz}
\usetikzlibrary{automata, positioning, arrows, calc}

\tikzset{
	->,
	>=stealth,
	node distance=3cm,
	every state/.style={thick, fill=gray!10},
	initial text=$ $,
}

\usepackage{tcolorbox}

%opening
\title{Paper assignement 5: Distance Vector Routing}
\author{Gabriel PEREIRA DE CARVALHO}
\date{Last modification: \today}

\begin{document}
	
	\maketitle
	
	\section*{Distance Vector Version 0}
	
	\begin{center}
		\begin{tabular}{ |c|c|c|c|c|c| } 
			\hline
			\textbf{$D_A$} & \textbf{Dest: B} & \textbf{Dest: C} &  \textbf{Dest: D} & \textbf{Dest: E} & \textbf{Dest: F} \\
			\hline
			\textbf{Via: B} & $\infty$ & $\infty$ & $\infty$ & $\infty$ & $\infty$ \\
			\hline 
			\textbf{Via: C} & $\infty$ & $\infty$ & $\infty$ & $\infty$ & $\infty$ \\
			\hline
			\textbf{Via: D} & $\infty$ & $\infty$ & $\infty$ & $\infty$ & $\infty$ \\
			\hline
			\textbf{Via: E} & $\infty$ & $\infty$ & $\infty$ & $\infty$ & $\infty$ \\
			\hline
			\textbf{Via: F} & $\infty$ & $\infty$ & $\infty$ & $\infty$ & $\infty$ \\
			\hline
		\end{tabular}
	\end{center}
	
	\section*{Distance Vector Version 1}
	
	\begin{itemize}
		\item The router boots and discovers its directly adjacent links.
	\end{itemize}
		
	\begin{center}
		\begin{tabular}{ |c|c|c|c|c|c| } 
			\hline
			\textbf{$D_A$} & \textbf{Dest: B} & \textbf{Dest: C} &  \textbf{Dest: D} & \textbf{Dest: E} & \textbf{Dest: F} \\
			\hline
			\textbf{Via: B} & $21$ & $\infty$ & $\infty$ & $\infty$ & $\infty$ \\
			\hline 
			\textbf{Via: C} & $\infty$ & $7$ & $\infty$ & $\infty$ & $\infty$ \\
			\hline
			\textbf{Via: D} & $\infty$ & $\infty$ & $\infty$ & $\infty$ & $\infty$ \\
			\hline
			\textbf{Via: E} & $\infty$ & $\infty$ & $\infty$ & $6$ & $\infty$ \\
			\hline
			\textbf{Via: F} & $\infty$ & $\infty$ & $\infty$ & $\infty$ & $5$ \\
			\hline
		\end{tabular}
	\end{center}
	
	\section*{Distance Vector Version 2}
	
	\begin{itemize}
		\item The router A receives the following DV messages:
		\begin{itemize}
			\item From B, containing \{(C,16), (D,8), (F,4)\}
			\item From C, containing \{(B,16), (D,3), (E,12)\}
			\item From E, containing \{(C,12), (D,9), (F,2)\}
			\item From F, containing \{(B,4), (D,14), (E,2)\}
		\end{itemize}
		\item We observe that A does not receive notifications for updates in D's routing table because routers A and D are not linked directly.
	\end{itemize}
	
	\begin{center}
		\begin{tabular}{ |c|c|c|c|c|c| } 
			\hline
			\textbf{$D_A$} & \textbf{Dest: B} & \textbf{Dest: C} &  \textbf{Dest: D} & \textbf{Dest: E} & \textbf{Dest: F} \\
			\hline
			\textbf{Via: B} & $21$ & $37$ & $29$ & $\infty$ & $25$ \\
			\hline 
			\textbf{Via: C} & $23$ & $7$ & $10$ & $19$ & $\infty$ \\
			\hline
			\textbf{Via: D} & $\infty$ & $\infty$ & $\infty$ & $\infty$ & $\infty$ \\
			\hline
			\textbf{Via: E} & $\infty$ & $18$ & $15$ & $6$ & $8$ \\
			\hline
			\textbf{Via: F} & $9$ & $\infty$ & $19$ & $7$ & $5$ \\
			\hline
		\end{tabular}
	\end{center}
	
	\section*{Distance Vector Version 3}
	
	\begin{itemize}
		\item The router A receives the following DV messages:
		\begin{itemize}
			\item From B, containing \{(C,11), (E,6)\}
			\item From C, containing \{(B,11), (F,12)\}
			\item From E, containing \{(B,6)\}
			\item From F, containing \{(C,12)\}
		\end{itemize}
	\end{itemize}
	
	\begin{center}
		\begin{tabular}{ |c|c|c|c|c|c| } 
			\hline
			\textbf{$D_A$} & \textbf{Dest: B} & \textbf{Dest: C} &  \textbf{Dest: D} & \textbf{Dest: E} & \textbf{Dest: F} \\
			\hline
			\textbf{Via: B} & $21$ & $32$ & $29$ & $27$ & $25$ \\
			\hline 
			\textbf{Via: C} & $18$ & $7$ & $10$ & $19$ & $19$ \\
			\hline
			\textbf{Via: D} & $\infty$ & $\infty$ & $\infty$ & $\infty$ & $\infty$ \\
			\hline
			\textbf{Via: E} & $12$ & $18$ & $15$ & $6$ & $8$ \\
			\hline
			\textbf{Via: F} & $9$ & $17$ & $16$ & $7$ & $5$ \\
			\hline
		\end{tabular}
	\end{center}
	
	After this iteration, the distance vector has converged and thus will not be modified in future iterations.

\end{document}